\documentclass[UKenglish, aspectratio = 169]{beamer}


\usetheme{OsloMet}
\usepackage{style}

\author[R. Chávez-Cáliz \& N. Bárcenas-Torres]
{Ricardo Esteban Chavez Cáliz \texorpdfstring{\\}{} }
\title{Stochastic Topology in the Curve Complex of a Surface}
\subtitle{Master in Science}


\begin{document}


\section{Overview}
\begin{frame}{Table of contents}
    \tableofcontents
\end{frame}

\section{The curve graph of a surface}
\SectionPage

\hidelogo
\begin{frame}{The curve graph of a surface}
\begin{definition}
The \textbf{curve graph} $\Gamma(S)$ of a surface $S$ is constructed with the following data:
\begin{itemize}
\item \textbf{Vertices}. There is a vertex in $\Gamma(S)$ for every isotopy class of essential, non-peripheral, simple, closed curves in $S$.
\item \textbf{Edges}. There is an edge between the corresponding vertices of isotopy classes $a$ and $b$ whenever $i(a,b)=0$.
\end{itemize}
\end{definition}
\end{frame}
\begin{frame}{The curve graph of a surface}

\end{frame}

\begin{frame}{The curve complex of a surface}
\begin{definition}
The \textbf{curve complex of the surface} $C(S)$ is defined to be the flag complex of the curve graph just defined.
\end{definition}
\end{frame}

\subsection{Definitions and notation}

\subsection{Properties of the curve graph}
\subsection{Rigidity in graphs}

\section{Rigidity in random graphs}
\SectionPage

\section{Computational experimentation}
\SectionPage

\section{Conclusions}
\SectionPage

%% Enable the logo in the lower right corner:
\showlogo

\subsection{Example}

\begin{frame}{Mathematics}
    \begin{example}
        The function \(\phi \colon \mathbb{R} \to \mathbb{R}\) given by \(\phi(x) = 2x\) is continuous at the point \(x = \alpha\),
        because if \(\epsilon > 0\) and \(x \in \mathbb{R}\) is such that \(\lvert x - \alpha \rvert < \delta = \frac{\epsilon}{2}\),
        then
        \begin{equation*}
            \lvert \phi(x) - \phi(\alpha)\rvert = 2\lvert x - \alpha \rvert < 2\delta = \epsilon.
        \end{equation*}
    \end{example}
\end{frame}

\begin{frame}{Highlighting}

    Some times it is useful to \alert{highlight} certain words in the text.

    \begin{alertblock}{Important message}
        If a lot of text should be \alert{highlighted}, it is a good idea to put it in a box.
    \end{alertblock}

    You can also highlight with the \structure{structure} colour.
\end{frame}

\section{Lists}

\begin{frame}{Lists}

    \begin{itemize}
        \item
        Bullet lists are marked with a yellow box.
    \end{itemize}

    \begin{enumerate}
        \item
        \label{enum:item}
        Numbered lists are marked with a black number inside a yellow box.
    \end{enumerate}

    \begin{description}
        \item[Description] highlights important words with blue text.
    \end{description}

    Items in numbered lists like \enumref{enum:item} can be referenced with a yellow box.

    \begin{example}
        \begin{itemize}
            \item
            Lists change colour after the environment.
        \end{itemize}
    \end{example}
\end{frame}

\section{Effects}

\begin{frame}{Effects}
    \begin{columns}[onlytextwidth]
        \begin{column}{0.49\textwidth}
            \begin{enumerate}[<+-|alert@+>]
                \item
                Effects that control

                \item
                when text is displayed

                \item
                are specified with <> and a list of slides.
            \end{enumerate}

            \begin{theorem}<2>
                This theorem is only visible on slide number 2.
            \end{theorem}
        \end{column}
        \begin{column}{0.49\textwidth}
            Use \textbf<2->{textblock} for arbitrary placement of objects.

            \pause
            \medskip

            It creates a box
            with the specified width (here in a percentage of the slide's width)
            and upper left corner at the specified coordinate (x, y)
            (here x is a percentage of width and y a percentage of height).
        \end{column}
    \end{columns}
    
    \begin{textblock}{0.3}(0.45, 0.55)
        \includegraphics<1, 3>[width = \textwidth]{example-image-a}
    \end{textblock}
\end{frame}

\section{References}

\begin{frame}[allowframebreaks]{References}
    \begin{thebibliography}{}

        % Article is the default.
        \setbeamertemplate{bibliography item}[book]

        \bibitem{Hartshorne1977}
        Hartshorne, R.
        \newblock \emph{Algebraic Geometry}.
        \newblock Springer-Verlag, 1977.

        \setbeamertemplate{bibliography item}[article]

        \bibitem{Helso2020}
        Helsø, M.
        \newblock \enquote{Rational quartic symmetroids}.
        \newblock \emph{Adv. Geom.}, 20(1):71--89, 2020.

        \setbeamertemplate{bibliography item}[online]

        \bibitem{HR2018}
        Helsø, M.\ and Ranestad, K.
        \newblock \emph{Rational quartic spectrahedra}, 2018.
        \newblock \url{https://arxiv.org/abs/1810.11235}

        \setbeamertemplate{bibliography item}[triangle]

        \bibitem{AM1969}
        Atiyah, M.\ and Macdonald, I.
        \newblock \emph{Introduction to commutative algebra}.
        \newblock Addison-Wesley Publishing Co., Reading, Mass.-London-Don
        Mills, Ont., 1969

        \setbeamertemplate{bibliography item}[text]

        \bibitem{Artin1966}
        Artin, M.
        \newblock \enquote{On isolated rational singularities of surfaces}.
        \newblock \emph{Amer. J. Math.}, 80(1):129--136, 1966.

    \end{thebibliography}
\end{frame}

\end{document}
